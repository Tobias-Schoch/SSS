%---------------
%╔═╗╔═╗╔╦╗╦ ╦╔═╗
%╚═╗║╣  ║ ║ ║╠═╝
%╚═╝╚═╝ ╩ ╚═╝╩  
%---------------

% language setup
\newcommand{\docLanguage}{ngerman}
%\newcommand{\docLanguage}{english}

% DOCUMENT SETUP
\documentclass[12pt, oneside, a4paper, \docLanguage]{report}
\usepackage[left=3cm, 
			right=2.5cm, 
			top=2.5cm, 
			bottom=2.5cm, 
			includehead, 
			includefoot]{geometry}

% line spacing
\usepackage{setspace}
\setstretch{1,25} % 15/12 --> 1.25

% encoding setup
% T1 font encoding for languages that use a latin alphabet
\usepackage[T1]{fontenc} 

% enhanced input encoding handling - utf8 for äÄüÜöÖß...
\usepackage[utf8]{inputenc}

%de­fines Adobe Times Ro­man as de­fault text font
\usepackage{mathptmx}
\usepackage{times} % needed for acronym package

%PDF linking package
\usepackage[hidelinks]{hyperref}


% Language Setup
\usepackage[\docLanguage]{babel}
% after babel - set chapter string
\AtBeginDocument{\renewcommand{\chaptername}{}}

% language specific bibliography style
\usepackage[numbers, square]{natbib}
%\setcitestyle{square,aysep={},yysep={;}}
\usepackage[fixlanguage]{babelbib}
\selectbiblanguage{\docLanguage}
% bliographystyle setup
% babel specific: babplain, babplai3, babalpha, babunsrt, bababbrv, bababbr3
\bibliographystyle{babunsrt}


% enumeration
\usepackage{enumitem}
% tabular extension tabularx
\usepackage{tabularx}

% math packages
\usepackage{amsmath}
\usepackage{nicefrac}
\usepackage{amsthm}
\usepackage{amsbsy}
\usepackage{amssymb}
\usepackage{amsfonts}
%\usepackage{MnSymbol}


%special characters
\usepackage{amssymb}
\usepackage{upgreek,textgreek}

% acronym package
\usepackage[printonlyused, footnote]{acronym}

% breakable text in \seqsplit{}
\usepackage{seqsplit}

% \textmu
\usepackage{textcomp}

% package provides a way to compile sections of a document using the same preamble as the main document
\usepackage{subfiles}

% driver-independent color extension - used by listings,tabularx
\usepackage[usenames,dvipsnames,table,xcdraw]{xcolor}

% -- SYNTAX HIGHLIGHTING --
\usepackage{listings}
%\input{cfgs/listings/listings_def_lang_bash-cmd.tex} % adds style BASH_CMD
%\input{cfgs/listings/listings_def_lang_bash-script.tex} % adds style BASH_SCRIPT
\input{cfgs/listings/listings_def_lang_latex.tex} % adds style LATEX
%\input{cfgs/listings/listings_def_lang_matlab.tex} % adds style MATLAB
\input{cfgs/listings/listings_def_lang_python.tex} % adds style PYTHON
%\input{cfgs/listings/listings_def_lang_c++.tex} % adds style CPP
%\input{cfgs/listings/listings_def_lang_c.tex} % adds style C
%\input{cfgs/listings/listings_def_lang_json.tex} % adds style JSON

% HEADLINE CFG
\usepackage{fancyhdr} % Headers and footers
\usepackage{lastpage}
\usepackage{ifthen}
\setlength{\headheight}{1.5cm}
%\pagestyle{fancy} % All pages have headers and footers
% override plain page style for \part, \chapter or 
% \maketitle, which implicit specifies plain page style
\input{cfgs/fancyhdr/fancyhdr_pagestyle_plain.tex}
% set list pagestyle
\input{cfgs/fancyhdr/fancyhdr_pagestyle_preface.tex}
% set default pagestyle
\input{cfgs/fancyhdr/fancyhdr_pagestyle_default_onepage.tex}
%\input{cfgs/fancyhdr/fancyhdr_pagestyle_default_twopage.tex}

\renewcommand{\chaptermark}[1]{\markright{#1}{}}
\renewcommand{\sectionmark}[1]{\markright{#1}{}}
\renewcommand{\headrulewidth}{0pt}
\renewcommand{\footrulewidth}{0pt}

% PICTURE CFG 
\usepackage{verbatim}
\usepackage{graphicx}
\usepackage{epstopdf}
\usepackage{caption}
\usepackage[list=true,listformat=simple]{subcaption}
% floating prevention packages
\usepackage{float}    % used with [H] positioning parameter
\usepackage{placeins} % \FloatBarrier 
% tikz packages
\usepackage{tikz}
\usepackage{standalone}
\usepackage{pgfplots}


% include only specified tex files - uncommend here
\includeonly{preface/cover,
             preface/abstract,
             preface/tableofcontents,
             preface/listoffigures,
             preface/listoftables,
             preface/lstlistoflistings,
             appendix/bibliography}

%-------------------
%╔═╗╔╦╗╦═╗╦╔╗╔╔═╗╔═╗
%╚═╗ ║ ╠╦╝║║║║║ ╦╚═╗
%╚═╝ ╩ ╩╚═╩╝╚╝╚═╝╚═╝
%-------------------
\newcommand{\strLecture}{Signale, Systeme und Sensoren}
\newcommand{\strDate}{\today}
\newcommand{\strAuthorA}{T. Schoch}
\newcommand{\strAuthorB}{L. Stratmann}
%\newcommand{\strAuthorC}{C. Author}
\newcommand{\strAuthorAEmail}{tobias.schoch@htwg-konstanz.de}
\newcommand{\strAuthorBEmail}{luca.stratmann@htwg-konstanz.de}
%\newcommand{\strAuthorCEmail}{cauthor@htwg-konstanz.de}
% Versuchsbeschreibung 
\newcommand{\strTopic}{Fourieranalyse und Akustik}
\newcommand{\strAbstract}{In dem dritten Versuch der Versuchsreihe werden wir die Fourieranalyse auf akustische Signale und Systeme in Form von Musik durch eine Mundharmonika und Rückkopplung anwenden. Die Signale werden auf einem Oszilloskop angezeigt.
\newline
In den beiden Teilen des Versuchs werden wir akustische Signale mittels der Python Bibliothek TekTDS2000 aufnemmen und abspeichern. Im Anschluss erfolgt mittels Python die Auswertung der Messdaten. 
\newline 
Dabei werden die Techniken der Fouriertransformation und des Bode Diagramms.
}
% hyperref customization
\hypersetup{
	pdftitle     = {\strTopic}, % title
	pdfsubject   = {\strLecture}, % subject of the document
	pdfauthor    = {\strAuthorA, \strAuthorB}, % author
	pdfkeywords  = {}, % list of keywords
	pdfcreator   = {}, % creator of the document
	pdfproducer  = {}, % producer of the document
	colorlinks   = false, % false: boxed links; true: colored links
	linkcolor    = red, % color of internal links (change box color with linkbordercolor)
    citecolor    = green, % color of links to bibliography
    filecolor    = magenta, % color of file links
    urlcolor     = cyan, % color of external links
	%bookmarks    = true, % show bookmarks bar?
	unicode	     = true, % non-Latin characters in Acrobat’s bookmarks
	pdftoolbar   = true, % show Acrobat’s toolbar?
	pdfmenubar   = true, % show Acrobat’s menu?
    pdffitwindow = false, % window fit to page when opened
	pdfnewwindow = true % links in new PDF window
}

%-----------------------------------------
% ╔╗ ╔═╗╔═╗╦╔╗╔  ╔╦╗╔═╗╔═╗╦ ╦╔╦╗╔═╗╔╗╔╔╦╗ 
% ╠╩╗║╣ ║ ╦║║║║   ║║║ ║║  ║ ║║║║║╣ ║║║ ║  
% ╚═╝╚═╝╚═╝╩╝╚╝  ═╩╝╚═╝╚═╝╚═╝╩ ╩╚═╝╝╚╝ ╩  
%-----------------------------------------

\begin{document}
\pagenumbering{Roman} 

\setcounter{section}{0}
\include{preface/cover}

\include{preface/abstract}
\clearpage

%
% TABLE OF CONTENTS
%
\include{preface/tableofcontents}

%
% Abbildungsverzeichnis
%
\include{preface/listoffigures}

%
% Tabellenverzeichnis
%
\include{preface/listoftables}

%
% Listingverzeichnis
%
\include{preface/lstlistoflistings}


%--------------------------
% ╔═╗╦ ╦╔═╗╔═╗╔╦╗╔═╗╦═╗╔═╗ 
% ║  ╠═╣╠═╣╠═╝ ║ ║╣ ╠╦╝╚═╗ 
% ╚═╝╩ ╩╩ ╩╩   ╩ ╚═╝╩╚═╚═╝ 
%--------------------------

\pagenumbering{arabic} 
\setcounter{page}{1} 
\pagestyle{default}
%
% CHAPTER Einleitung
%
\chapter{Einleitung}
\label{chap:EINL}

\cite{Franz2016n}
\cite{Franz2016e}

%
% CHAPTER Versuch 1
%
\chapter{Versuch 1}
\label{chap:VERSUCH_1}

\section{Fragestellung, Messprinzip, Aufbau, Messmittel}
\label{chap:VERSUCH_1_FRAGESTELLUNG}
Im ersten Versuch der Versuchsreihe "Fourieranalyse und Akustik" werden wir die Fourieranalyse auf akustische Signale und Systeme anwenden.
Dazu werden wir mittels einer Mundharmonika in ein Mikrofon blasen, dass wiederum an ein Oszillskop angeschlossen ist und die Spannungen anzeigt.
Dabei ist es wichtig zu beachten, dass auf dem Oszilloskop mehrere Perioden abgegebildet werden. Das Triggerlevel soll so im Oszilloskop eingestellt werden, dass das Signal nur bei genügend hoher Amplitude eingestellt wird.
Die Triggerung sollte dabei auch im Single Sequence Modus eingestellt werden um statische Daten zu erhalten die auch dem tatsächlichen Spannungswert entsprechen.
Nachdem dies erledigt ist, haben wir ein Pythonskript geschrieben task1.1.py um das Signal aus dem Oszilloskop bei der Einstellung der Triggerung "Single Sequenz" auszulesen.
Dies wird mittels der Toolbox TekTDS2000 von M. Miller in eine .csv Datei gespeichert welche zum Beispiel mit Excel geöffnet werden kann.
\newpage
So sieht der Versuchsaufbau des ersten Versuches aus. Dabei wird das Mikrofon an den Channel 1 des Oszilloskopes angeschlossen. Nach der Kalibrierung und der Einstellung der künstlichen Sinuskurve haben wir mit dem Pythonskript das ausgegebene Signal des Oszilloskopes in eine .csv Datei eingelesen.
Danach hatten wir die Aufgabe folgende Werte zu berechnen.~\par
\begin{itemize}
	\item Grundperiode (in ms)
	\item Grundfrequenz (in Hz)
	\item Signaldauer (in s)
	\item Abtastfrequenz (in Hz)
	\item Signallänge M (Anzahl der Abtastzeitpunkte)
	\item Abtastintervall $\Delta$t (in s)
\end{itemize}
\begin{figure}[H]
	\centering\small
	\includegraphics[width=0.8\textwidth]{media/aufbau2.jpg}
	\caption{Versuchsaufbau Teil 1}
	\label{img:Versuchsaufbau Teil 1}
\end{figure}
\newpage
Im Anschluss mussten wir mithilfe der Funktion numpy.fft.fft() die Fouriertransformation des Signals berechnen. Daraus sollten wir das Amplitudenspektrum bestimmen und graphisch darstellen.
Dabei ist zu beachten, dass die x-Achse mit der Frequenz nicht in Hertz sondern in der Anzahl Schwingungen innerhlab der gesamten Signaldauer definiert ist..
Die Frequenz f in Hertz lässt sich jedoch folgendermaße berechnen.
\\~\\
f = $\dfrac{n}{M * \Delta t}$
\\~\\
Als letzten Teil der ersten Aufgabe sollen wir die Grundfrequenz im Spektrum berechnen und damit die Frequenz in Hertz identifizieren.
Folgende Materialien wurden benötigt: ~\par
\begin{itemize}
	\item Oszilloskop
	\item Mikrofon
	\item Mundharmonika
	\item Signalkabel
	\item Python auf einem Computer
\end{itemize}
\newpage
\section{Messwerte}
\label{chap:VERSUCH_1_MESSWERTE}
Auf dem linken Bild sieht man die Darstellung auf dem Oszilloskop. Diese wurde durch ein Programm auf dem Labor Computer eingelesen.
Durch einen Pythoncode den wir aus der Toolbox TekTDS2000 erhalten haben, konnten wir den Spannungsverlauf aus dem Oszilloskop einlesen und in eine .csv Datei schreiben. Dieses Format wird auch häufig für Excel Dateien verwendet.
Diese Datei haben wir mittels Numpy und der Funktion np.loadfromtxt ausgelesen.
Die Ergebnisse wurden in die 2 verschiedene Funktionen x und y geschrieben.
Danach wurden diese mittels matplotlib graphisch dargestellt.
\begin{figure}[H]
\centering
\begin{subfigure}{.5\textwidth}
  \centering
  \includegraphics[width=0.9\linewidth]{../data/img/100.png}
  \caption{Die Ausgabe des Oszilloskopes}
  \label{fig:Die Ausgabe des Oszilloskopes}
\end{subfigure}%
\begin{subfigure}{.5\textwidth}
  \centering
  \includegraphics[width=0.9\linewidth]{../data/img/gross100.png}
  \caption{Die Auswertung der .csv Datei durch Python}
  \label{fig:Die Auswertung der .csv Datei durch Python}
\end{subfigure}
\caption{Einlesen des Signals in .csv und duch das Oszilloskop}
\label{fig:Einlesen des Signals in .csv und duch das Oszilloskop}
\end{figure}

Im folgenden sind die Ergebnisse der Berechnungen aufzufinden:
\begin{table}[H]
	\centering\small
	\begin{tabular}{|c|c|}
	\hline
	Plot & Wert \\
	\hline
	Grundperiode & 0.001275s \\
	\hline
	Grundfrequenz & 784.31Hz \\
	\hline
	Signaldauer &  \\
	\hline
	Abtastfrequenz &  \\
	\hline
	Signallänge &  \\
	\hline
	Abtastintervall & 
	\hline
	\end{tabular}
	\caption{Mittelwert, Hexwert und Standardabweichung für die Intensivität des Lichteinfalls}
	\label{fig:VERSUCH_4_MESSWERTE}
\end{table}

\section{Auswertung}

\label{chap:VERSUCH_1_AUSWERTUNG}

\section{Interpretation}
\label{chap:VERSUCH_1_INTERPRETATION}

%
% CHAPTER Versuch 2
%
\chapter{Versuch 2}
\label{chap:VERSUCH_2}

\section{Fragestellung, Messprinzip, Aufbau, Messmittel}
\label{chap:VERSUCH_2_FRAGESTELLUNG}

\section{Messwerte}
\label{chap:VERSUCH_2_MESSWERTE}

\section{Auswertung}
\label{chap:VERSUCH_2_AUSWERTUNG}

\section{Interpretation}
\label{chap:VERSUCH_2_INTERPRETATION}

%
% CHAPTER Anhang
%
\renewcommand\thesection{A.\arabic{section}}
\renewcommand\thesubsection{\thesection.\arabic{subsection}}

\chapter*{Anhang}
\label{chap:APPENDIX}
\addcontentsline{toc}{chapter}{Anhang}
%\setcounter{chapter}{0}
\addtocounter{chapter}{1}
\setcounter{section}{0}

\section{Quellcode}
\label{chap:APPENDIX_SOURCECODE}

\subsection{Quellcode Versuch 1}
\label{chap:APPENDIX_SOURCECODE_V1}

\subsection{Quellcode Versuch 2}
\label{chap:APPENDIX_SOURCECODE_V2}

\subsection{Quellcode Versuch 3}
\label{chap:APPENDIX_SOURCECODE_V3}

\subsection{Quellcode Versuch 4}
\label{chap:APPENDIX_SOURCECODE_V4}


\section{Messergebnisse}
\label{chap:APPENDIX_MEASUREMENT_SOURCE}

%
% Literaturverzeichnis
%
\include{appendix/bibliography}

\end{document}
%------------------------------------
% ╔═╗╔╗╔╔╦╗  ╔╦╗╔═╗╔═╗╦ ╦╔╦╗╔═╗╔╗╔╔╦╗
% ║╣ ║║║ ║║   ║║║ ║║  ║ ║║║║║╣ ║║║ ║ 
% ╚═╝╝╚╝═╩╝  ═╩╝╚═╝╚═╝╚═╝╩ ╩╚═╝╝╚╝ ╩ 
%------------------------------------