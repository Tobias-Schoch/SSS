%---------------
%╔═╗╔═╗╔╦╗╦ ╦╔═╗
%╚═╗║╣  ║ ║ ║╠═╝
%╚═╝╚═╝ ╩ ╚═╝╩  
%---------------

% language setup
\newcommand{\docLanguage}{ngerman}
%\newcommand{\docLanguage}{english}

% DOCUMENT SETUP
\documentclass[12pt, oneside, a4paper, \docLanguage]{report}
\usepackage[left=3cm, 
			right=2.5cm, 
			top=2.5cm, 
			bottom=2.5cm, 
			includehead, 
			includefoot]{geometry}

% line spacing
\usepackage{setspace}
\setstretch{1,25} % 15/12 --> 1.25

% encoding setup
% T1 font encoding for languages that use a latin alphabet
\usepackage[T1]{fontenc} 

% enhanced input encoding handling - utf8 for äÄüÜöÖß...
\usepackage[utf8]{inputenc}

%de­fines Adobe Times Ro­man as de­fault text font
\usepackage{mathptmx}
\usepackage{times} % needed for acronym package

%PDF linking package
\usepackage[hidelinks]{hyperref}


% Language Setup
\usepackage[\docLanguage]{babel}
% after babel - set chapter string
\AtBeginDocument{\renewcommand{\chaptername}{}}

% language specific bibliography style
\usepackage[numbers, square]{natbib}
%\setcitestyle{square,aysep={},yysep={;}}
\usepackage[fixlanguage]{babelbib}
\selectbiblanguage{\docLanguage}
% bliographystyle setup
% babel specific: babplain, babplai3, babalpha, babunsrt, bababbrv, bababbr3
\bibliographystyle{babunsrt}


% enumeration
\usepackage{enumitem}
% tabular extension tabularx
\usepackage{tabularx}

% math packages
\usepackage{amsmath}
\usepackage{nicefrac}
\usepackage{amsthm}
\usepackage{amsbsy}
\usepackage{amssymb}
\usepackage{amsfonts}
%\usepackage{MnSymbol}


%special characters
\usepackage{amssymb}
\usepackage{upgreek,textgreek}

% acronym package
\usepackage[printonlyused, footnote]{acronym}

% breakable text in \seqsplit{}
\usepackage{seqsplit}

% \textmu
\usepackage{textcomp}

% package provides a way to compile sections of a document using the same preamble as the main document
\usepackage{subfiles}

% driver-independent color extension - used by listings,tabularx
\usepackage[usenames,dvipsnames,table,xcdraw]{xcolor}

% -- SYNTAX HIGHLIGHTING --
\usepackage{listings}
%\input{cfgs/listings/listings_def_lang_bash-cmd.tex} % adds style BASH_CMD
%\input{cfgs/listings/listings_def_lang_bash-script.tex} % adds style BASH_SCRIPT
\input{cfgs/listings/listings_def_lang_latex.tex} % adds style LATEX
%\input{cfgs/listings/listings_def_lang_matlab.tex} % adds style MATLAB
\input{cfgs/listings/listings_def_lang_python.tex} % adds style PYTHON
%\input{cfgs/listings/listings_def_lang_c++.tex} % adds style CPP
%\input{cfgs/listings/listings_def_lang_c.tex} % adds style C
%\input{cfgs/listings/listings_def_lang_json.tex} % adds style JSON

% HEADLINE CFG
\usepackage{fancyhdr} % Headers and footers
\usepackage{lastpage}
\usepackage{ifthen}
\setlength{\headheight}{1.5cm}
%\pagestyle{fancy} % All pages have headers and footers
% override plain page style for \part, \chapter or 
% \maketitle, which implicit specifies plain page style
\input{cfgs/fancyhdr/fancyhdr_pagestyle_plain.tex}
% set list pagestyle
\input{cfgs/fancyhdr/fancyhdr_pagestyle_preface.tex}
% set default pagestyle
\input{cfgs/fancyhdr/fancyhdr_pagestyle_default_onepage.tex}
%\input{cfgs/fancyhdr/fancyhdr_pagestyle_default_twopage.tex}

\renewcommand{\chaptermark}[1]{\markright{#1}{}}
\renewcommand{\sectionmark}[1]{\markright{#1}{}}
\renewcommand{\headrulewidth}{0pt}
\renewcommand{\footrulewidth}{0pt}

% PICTURE CFG 
\usepackage{verbatim}
\usepackage{graphicx}
\usepackage{epstopdf}
\usepackage{caption}
\usepackage[list=true,listformat=simple]{subcaption}
% floating prevention packages
\usepackage{float}    % used with [H] positioning parameter
\usepackage{placeins} % \FloatBarrier 
% tikz packages
\usepackage{tikz}
\usepackage{standalone}
\usepackage{pgfplots}


% include only specified tex files - uncommend here
\includeonly{preface/cover,
             preface/abstract,
             preface/tableofcontents,
             preface/listoffigures,
             preface/listoftables,
             preface/lstlistoflistings,
             appendix/bibliography}

%-------------------
%╔═╗╔╦╗╦═╗╦╔╗╔╔═╗╔═╗
%╚═╗ ║ ╠╦╝║║║║║ ╦╚═╗
%╚═╝ ╩ ╩╚═╩╝╚╝╚═╝╚═╝
%-------------------
\newcommand{\strLecture}{Signale, Systeme und Sensoren}
\newcommand{\strDate}{\today}
\newcommand{\strAuthorA}{Tobias Schoch}
\newcommand{\strAuthorB}{Luca Strattmann}
%\newcommand{\strAuthorC}{C. Author}
\newcommand{\strAuthorAEmail}{tobias.schoch@htwg-konstanz.de}
\newcommand{\strAuthorBEmail}{luca.strattmann@htwg-konstanz.de}
%\newcommand{\strAuthorCEmail}{cauthor@htwg-konstanz.de}
% Versuchsbeschreibung 
\newcommand{\strTopic}{Aufbau eines einfachen Spracherkenners}
\newcommand{\strAbstract}{Zusammenfassung etwa 100 Worte.}
% hyperref customization
\hypersetup{
	pdftitle     = {\strTopic}, % title
	pdfsubject   = {\strLecture}, % subject of the document
	pdfauthor    = {\strAuthorA, \strAuthorB}, % author
	pdfkeywords  = {}, % list of keywords
	pdfcreator   = {}, % creator of the document
	pdfproducer  = {}, % producer of the document
	colorlinks   = false, % false: boxed links; true: colored links
	linkcolor    = red, % color of internal links (change box color with linkbordercolor)
    citecolor    = green, % color of links to bibliography
    filecolor    = magenta, % color of file links
    urlcolor     = cyan, % color of external links
	%bookmarks    = true, % show bookmarks bar?
	unicode	     = true, % non-Latin characters in Acrobat’s bookmarks
	pdftoolbar   = true, % show Acrobat’s toolbar?
	pdfmenubar   = true, % show Acrobat’s menu?
    pdffitwindow = false, % window fit to page when opened
	pdfnewwindow = true % links in new PDF window
}

%-----------------------------------------
% ╔╗ ╔═╗╔═╗╦╔╗╔  ╔╦╗╔═╗╔═╗╦ ╦╔╦╗╔═╗╔╗╔╔╦╗ 
% ╠╩╗║╣ ║ ╦║║║║   ║║║ ║║  ║ ║║║║║╣ ║║║ ║  
% ╚═╝╚═╝╚═╝╩╝╚╝  ═╩╝╚═╝╚═╝╚═╝╩ ╩╚═╝╝╚╝ ╩  
%-----------------------------------------

\begin{document}
\pagenumbering{Roman} 

\setcounter{section}{0}
\include{preface/cover}

\include{preface/abstract}
\clearpage

%
% TABLE OF CONTENTS
%
\include{preface/tableofcontents}

%
% Abbildungsverzeichnis
%
\include{preface/listoffigures}

%
% Tabellenverzeichnis
%
\include{preface/listoftables}

%
% Listingverzeichnis
%
\include{preface/lstlistoflistings}


%--------------------------
% ╔═╗╦ ╦╔═╗╔═╗╔╦╗╔═╗╦═╗╔═╗ 
% ║  ╠═╣╠═╣╠═╝ ║ ║╣ ╠╦╝╚═╗ 
% ╚═╝╩ ╩╩ ╩╩   ╩ ╚═╝╩╚═╚═╝ 
%--------------------------

\pagenumbering{arabic} 
\setcounter{page}{1} 
\pagestyle{default}
%
% CHAPTER Einleitung
%
\chapter{Einleitung}
\label{chap:EINL}

\cite{Franz2016n}
\cite{Franz2016e}

%
% CHAPTER Versuch 1
%
\chapter{Versuch 1}
\label{chap:VERSUCH_1}

\section{Fragestellung, Messprinzip, Aufbau, Messmittel}
\label{chap:VERSUCH_1_FRAGESTELLUNG}

\textbf{Fragestellung:}
\newline
In dem ersten Teil des Versuchs 'Aufbau eines einfachen Spracherkenners' werden wir einen Spracherkenner bauen.
Zudem erweitern wir den Spracherkenner mit einer Triggerfunktion.
Mit der Aufnahme bestimmen wir dessen Amplitudenspektrum.
Im Anschluss werden die Methode des Windowing anwenden.
\newline
So werden wir eine Testaufnahme machen um diese anschließend auszuwerten. 
\newline
Auf die entstandenen Numpy-Dateien wenden wir Windowing und das Amplitudenspektrum an.
\newline
\newline
\textbf{Messprinzip:}
\newline
Im ersten Versuch starten wir mit einem Pythonskript. Das Mikrofon wird mit einem Klinkenstecker direkt an die Soundkarte des Computers verbunden.
Auf den Computern im Labor, ist das Paket PyAudio bereits in den IDE's integriert.
\newline
Mit dem geschriebenen Pythonskript können wir nun akustische Signale aufnehmen. 
Über das Objekt audiorecorder haben wir Zugriff auf die Aufnahmefunktion der Soundkarte.
\newline
Das Signal speichern wir anschließend mittels numpy.save().
Im Anschluss sollen wir eine beliebige Spracheingabe aufnehmen und diese in einem Diagramm darstellen.
\newline
\newline
Im Anschluss sollen wir das Aufnahmeprogramm um eine Triggerfunktion erweitern, welche die Aufnahme erst ab einem gewissen Lautstärkepegel starten lässt.
\newline
So können wir sicher stellen, dass alle Aufnahmen den selben Startpunkt besitzen.
\newpage
Das Signal soll eine Dauer von einer Sekunde haben und die fehlenden Samples mit Nullen augefüllt werden.
\newline
\newline
Mit dem Code Aus dem dritten Versuch können wir mit der Aufnahme das Amplitudenspektrum bestimmen.
Dies stellen wir grapisch dar.
Danach implementieren die Methode des Windowing. 
\newline
Diese werden wir jeweils in einer Länge von 512 Samples darstellen.
Die einzelnen Windows werden wir mit der Gaußschen Fensterfunktion multiplizieren  die eine Fensterbreite von der Standartabweichung 4 hat.
\newline
Den ersten Versuch werden wir mit dem Amplitudenspektrum erneut überprüfen und so das Spektrum aus der letzten Aufgabe auf Korrektheit überprüfen.
\newline
\newline
\textbf{Aufbau:}
\newline
Das Mikrofon wird durch einen Klinkenstecker direkt an die Soundkarte des Computers verbunden.
Durch ein Pythonskript können wir nun Sprachaufnahmen machen.
\begin{figure}[H]
	\centering
	\includegraphics[width=.5\linewidth]{media/aufbau.jpg}
	\caption{Aufnahme mit Triggerfunktion des Wortes rechts}
	\label{img:Aufnahme mit Triggerfunktion des Wortes rechts}
\end{figure}
\newpage
\textbf{Messmittel:}
\item[NichtnummerierteAufzahlung]~\par
   \begin{itemize}
      \item Ein Mikrofon
      \item Ein Computer mit einer Python IDE
   \end{itemize}
\newpage
\section{Messwerte}
\label{chap:VERSUCH_1_MESSWERTE}
Mit einem Mikrofon das an den Computer angeschlossen ist, sehen wir die mit dem Pythonskript aufgenommene Sprachaufnahme des Wortes 'Test', welche mit Python visualisiert wurde.
In der linken Darstellung ist die gesamte Dauer des empfangenen Signals.
Im rechten Bild ist eine verkürzte Darstellung der Sprachaufnahme um das Signal besser zu erkennen.
\begin{figure}[hbt!]
\centering
	\begin{subfigure}{.5\textwidth}
  		\centering
 		 \includegraphics[width=.95\linewidth]{../data/img/testamp.png}
  		\caption{Sprachaufnahme des Wortes Test}
 		 \label{fig:sub1}
	\end{subfigure}%
	\begin{subfigure}{.5\textwidth}
  		\centering
 		 \includegraphics[width=.95\linewidth]{../data/img/testamp2.png}
  		\caption{Dieselbe Sprachaufnahme mit kurzer Zeitachse}
  		\label{fig:sub2}
	\end{subfigure}
	\caption{Die aufgenommene Sprachaufnahme visualisiert mit Python}
	\label{fig:test}
\end{figure}
\newline
Im folgenden sieht man die Sprachaufnahme des Wortes 'Rechts' mit der Triggerfunktion.
\begin{figure}[H]
	\centering
	\includegraphics[width=.6\linewidth]{../data/img/rechtsamp.png}
	\caption{Aufnahme mit Triggerfunktion des Wortes rechts}
	\label{img:Aufnahme mit Triggerfunktion des Wortes rechts}
\end{figure}


\section{Auswertung}
\label{chap:VERSUCH_1_AUSWERTUNG}
\begin{figure}[H]
\centering
	\begin{subfigure}{.5\textwidth}
  		\centering
 		 \includegraphics[width=.95\linewidth]{../data/img/testspektrum1.png}
  		\caption{Amplitudenspektrum der Sprachaufnahme}
 		 \label{fig:sub1}
	\end{subfigure}%
	\begin{subfigure}{.5\textwidth}
  		\centering
 		 \includegraphics[width=.95\linewidth]{../data/img/testspektrum3.png}
  		\caption{Amplitudenspektrum mit geringerer Frequenz}
  		\label{fig:sub2}
	\end{subfigure}
	\caption{Das Amplitudenspektrum mit der dazugehörigen Frequenz}
	\label{fig:test}
\end{figure}
\newline
Im folgenden sieht man die Sprachaufnahme des Wortes 'Rechts' mit der Triggerfunktion.
\begin{figure}[H]
	\centering
	\includegraphics[width=.8\linewidth]{../data/img/gauss.png}
	\caption{Aufnahme mit Triggerfunktion des Wortes rechts}
	\label{img:Aufnahme mit Triggerfunktion des Wortes rechts}
\end{figure}
Im folgenden sieht man die Sprachaufnahme des Wortes 'Rechts' mit der Triggerfunktion.
\begin{figure}[H]
	\centering
	\includegraphics[width=.8\linewidth]{../data/img/AlleRichtig.png}
	\caption{Aufnahme mit Triggerfunktion des Wortes rechts}
	\label{img:Aufnahme mit Triggerfunktion des Wortes rechts}
\end{figure}
Im folgenden sieht man die Sprachaufnahme des Wortes 'Rechts' mit der Triggerfunktion.
\begin{figure}[H]
	\centering
	\includegraphics[width=.8\linewidth]{../data/img/window.png}
	\caption{Aufnahme mit Triggerfunktion des Wortes rechts}
	\label{img:Aufnahme mit Triggerfunktion des Wortes rechts}
\end{figure}
\newpage
\section{Interpretation}
\label{chap:VERSUCH_1_INTERPRETATION}

%
% CHAPTER Versuch 2
%
\chapter{Versuch 2}
\label{chap:VERSUCH_2}

\section{Fragestellung, Messprinzip, Aufbau, Messmittel}
\label{chap:VERSUCH_2_FRAGESTELLUNG}

\section{Messwerte}
\label{chap:VERSUCH_2_MESSWERTE}

\section{Auswertung}
\label{chap:VERSUCH_2_AUSWERTUNG}

\section{Interpretation}
\label{chap:VERSUCH_2_INTERPRETATION}

%
% CHAPTER Anhang
%
\renewcommand\thesection{A.\arabic{section}}
\renewcommand\thesubsection{\thesection.\arabic{subsection}}

\chapter*{Anhang}
\label{chap:APPENDIX}
\addcontentsline{toc}{chapter}{Anhang}
%\setcounter{chapter}{0}
\addtocounter{chapter}{1}
\setcounter{section}{0}

\section{Quellcode}
\label{chap:APPENDIX_SOURCECODE}

\subsection{Quellcode Versuch 1}
\label{chap:APPENDIX_SOURCECODE_V1}
\lstinputlisting[style=PYTHON, frame=single, caption=Einlesen der Sprachaufnahme und ablegen des Signals in eine Numpy Datei, captionpos=b, label=lst:APPENDIX_SOURCECODE_AREA1]{../task1.1.py}
\newpage
\lstinputlisting[style=PYTHON, frame=single, caption=Einlesen einer Sprachaufnahme mit Aktivierung durch Triggerung, captionpos=b, label=lst:APPENDIX_SOURCECODE_AREA2]{../task1.2.py}
\newpage
\lstinputlisting[style=PYTHON, frame=single, caption=Amplitudenspektrum und Ausgabe von Plots, captionpos=b, label=lst:APPENDIX_SOURCECODE_AREA3]{../task1.3.py}
\newpage
\lstinputlisting[style=PYTHON, frame=single, caption=Windowing und Ausgabe von Plots bzw. Windows, captionpos=b, label=lst:APPENDIX_SOURCECODE_AREA4]{../task1.4.py}
\newpage
\subsection{Quellcode Versuch 2}
\label{chap:APPENDIX_SOURCECODE_V2}
\lstinputlisting[style=PYTHON, frame=single, caption=Windowing und Mittelung der Spektren, captionpos=b, label=lst:APPENDIX_SOURCECODE_AREA5]{../task2.1.py}
\newpage
\lstinputlisting[style=PYTHON, frame=single, caption=Windowing und Bravais-Pearson Methode, captionpos=b, label=lst:APPENDIX_SOURCECODE_AREA6]{../task2.2.py}
\newpage
\lstinputlisting[style=PYTHON, frame=single, caption=Bravais-Pearson Methode mit Ausgabe der Korrelation, captionpos=b, label=lst:APPENDIX_SOURCECODE_AREA7]{../task2.3.py}

%
% Literaturverzeichnis
%
\include{appendix/bibliography}

\end{document}
%------------------------------------
% ╔═╗╔╗╔╔╦╗  ╔╦╗╔═╗╔═╗╦ ╦╔╦╗╔═╗╔╗╔╔╦╗
% ║╣ ║║║ ║║   ║║║ ║║  ║ ║║║║║╣ ║║║ ║ 
% ╚═╝╝╚╝═╩╝  ═╩╝╚═╝╚═╝╚═╝╩ ╩╚═╝╝╚╝ ╩ 
%------------------------------------