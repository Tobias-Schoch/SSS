%---------------
%╔═╗╔═╗╔╦╗╦ ╦╔═╗
%╚═╗║╣  ║ ║ ║╠═╝
%╚═╝╚═╝ ╩ ╚═╝╩  
%---------------

% language setup
\newcommand{\docLanguage}{ngerman}
%\newcommand{\docLanguage}{english}

% DOCUMENT SETUP
\documentclass[12pt, oneside, a4paper, \docLanguage]{report}
\usepackage[left=3cm, 
			right=2.5cm, 
			top=2.5cm, 
			bottom=2.5cm, 
			includehead, 
			includefoot]{geometry}

% line spacing
\usepackage{setspace}
\setstretch{1,25} % 15/12 --> 1.25

% encoding setup
% T1 font encoding for languages that use a latin alphabet
\usepackage[T1]{fontenc} 

% enhanced input encoding handling - utf8 for äÄüÜöÖß...
\usepackage[utf8]{inputenc}

%de­fines Adobe Times Ro­man as de­fault text font
\usepackage{mathptmx}
\usepackage{times} % needed for acronym package

%PDF linking package
\usepackage[hidelinks]{hyperref}


% Language Setup
\usepackage[\docLanguage]{babel}
% after babel - set chapter string
\AtBeginDocument{\renewcommand{\chaptername}{}}

% language specific bibliography style
\usepackage[numbers, square]{natbib}
%\setcitestyle{square,aysep={},yysep={;}}
\usepackage[fixlanguage]{babelbib}
\selectbiblanguage{\docLanguage}
% bliographystyle setup
% babel specific: babplain, babplai3, babalpha, babunsrt, bababbrv, bababbr3
\bibliographystyle{babunsrt}


% enumeration
\usepackage{enumitem}
% tabular extension tabularx
\usepackage{tabularx}

% math packages
\usepackage{amsmath}
\usepackage{nicefrac}
\usepackage{amsthm}
\usepackage{amsbsy}
\usepackage{amssymb}
\usepackage{amsfonts}
%\usepackage{MnSymbol}


%special characters
\usepackage{amssymb}
\usepackage{upgreek,textgreek}

% acronym package
\usepackage[printonlyused, footnote]{acronym}

% breakable text in \seqsplit{}
\usepackage{seqsplit}

% \textmu
\usepackage{textcomp}

% package provides a way to compile sections of a document using the same preamble as the main document
\usepackage{subfiles}

% driver-independent color extension - used by listings,tabularx
\usepackage[usenames,dvipsnames,table,xcdraw]{xcolor}

% -- SYNTAX HIGHLIGHTING --
\usepackage{listings}
%\input{cfgs/listings/listings_def_lang_bash-cmd.tex} % adds style BASH_CMD
%\input{cfgs/listings/listings_def_lang_bash-script.tex} % adds style BASH_SCRIPT
\input{cfgs/listings/listings_def_lang_latex.tex} % adds style LATEX
%\input{cfgs/listings/listings_def_lang_matlab.tex} % adds style MATLAB
\input{cfgs/listings/listings_def_lang_python.tex} % adds style PYTHON
%\input{cfgs/listings/listings_def_lang_c++.tex} % adds style CPP
%\input{cfgs/listings/listings_def_lang_c.tex} % adds style C
%\input{cfgs/listings/listings_def_lang_json.tex} % adds style JSON

% HEADLINE CFG
\usepackage{fancyhdr} % Headers and footers
\usepackage{lastpage}
\usepackage{ifthen}
\setlength{\headheight}{1.5cm}
%\pagestyle{fancy} % All pages have headers and footers
% override plain page style for \part, \chapter or 
% \maketitle, which implicit specifies plain page style
\input{cfgs/fancyhdr/fancyhdr_pagestyle_plain.tex}
% set list pagestyle
\input{cfgs/fancyhdr/fancyhdr_pagestyle_preface.tex}
% set default pagestyle
\input{cfgs/fancyhdr/fancyhdr_pagestyle_default_onepage.tex}
%\input{cfgs/fancyhdr/fancyhdr_pagestyle_default_twopage.tex}

\renewcommand{\chaptermark}[1]{\markright{#1}{}}
\renewcommand{\sectionmark}[1]{\markright{#1}{}}
\renewcommand{\headrulewidth}{0pt}
\renewcommand{\footrulewidth}{0pt}

% PICTURE CFG 
\usepackage{verbatim}
\usepackage{graphicx}
\usepackage{epstopdf}
\usepackage{caption}
\usepackage[list=true,listformat=simple]{subcaption}
% floating prevention packages
\usepackage{float}    % used with [H] positioning parameter
\usepackage{placeins} % \FloatBarrier 
% tikz packages
\usepackage{tikz}
\usepackage{standalone}
\usepackage{pgfplots}


% include only specified tex files - uncommend here
\includeonly{preface/cover,
             preface/abstract,
             preface/tableofcontents,
             preface/listoffigures,
             preface/listoftables,
             preface/lstlistoflistings,
             appendix/bibliography}

%-------------------
%╔═╗╔╦╗╦═╗╦╔╗╔╔═╗╔═╗
%╚═╗ ║ ╠╦╝║║║║║ ╦╚═╗
%╚═╝ ╩ ╩╚═╩╝╚╝╚═╝╚═╝
%-------------------
\newcommand{\strLecture}{Signale, Systeme und Sensoren}
\newcommand{\strDate}{\today}
\newcommand{\strAuthorA}{Tobias Schoch}
\newcommand{\strAuthorB}{Luca Strattmann}
%\newcommand{\strAuthorC}{C. Author}
\newcommand{\strAuthorAEmail}{tobias.schoch@htwg-konstanz.de}
\newcommand{\strAuthorBEmail}{luca.strattmann@htwg-konstanz.de}
%\newcommand{\strAuthorCEmail}{cauthor@htwg-konstanz.de}
% Versuchsbeschreibung 
\newcommand{\strTopic}{Aufbau eines einfachen Spracherkenners}
\newcommand{\strAbstract}{In dem vierten Versuch der Versuchsreihe mit dem Titel  \textit{Aufbau eines einfachen Spracherkenners}, geht es um einen Spracherkenner und dessen Auswertung.
\newline
So werden wir einen einfachen Spracherkenner aufbauen, der beispielsweise zur Steuerung eines Staplers in einem Hochregallager dienen könnte.
\newline
\newline
Es reichen hierzu die Erkennung von vier einfachen Befehlen (Hoch, Tief, Links, Rechts). Der Aufbau  des Spracherkenners folgt dem in der Vorlesung beschriebenen Prinzip des Prototypen Klassifikators. 
\newline
Wir werden die Methoden des Windowing, der Triggerung, des Amplitudenspektrums und der Korrelationskoeffizienten mit Bravais-Pearson anwenden.
\newline
Das und vieles Spannende mehr folgt auf den kommenden Seiten.}
% hyperref customization
\hypersetup{
	pdftitle     = {\strTopic}, % title
	pdfsubject   = {\strLecture}, % subject of the document
	pdfauthor    = {\strAuthorA, \strAuthorB}, % author
	pdfkeywords  = {}, % list of keywords
	pdfcreator   = {}, % creator of the document
	pdfproducer  = {}, % producer of the document
	colorlinks   = false, % false: boxed links; true: colored links
	linkcolor    = red, % color of internal links (change box color with linkbordercolor)
    citecolor    = green, % color of links to bibliography
    filecolor    = magenta, % color of file links
    urlcolor     = cyan, % color of external links
	%bookmarks    = true, % show bookmarks bar?
	unicode	     = true, % non-Latin characters in Acrobat’s bookmarks
	pdftoolbar   = true, % show Acrobat’s toolbar?
	pdfmenubar   = true, % show Acrobat’s menu?
    pdffitwindow = false, % window fit to page when opened
	pdfnewwindow = true % links in new PDF window
}

%-----------------------------------------
% ╔╗ ╔═╗╔═╗╦╔╗╔  ╔╦╗╔═╗╔═╗╦ ╦╔╦╗╔═╗╔╗╔╔╦╗ 
% ╠╩╗║╣ ║ ╦║║║║   ║║║ ║║  ║ ║║║║║╣ ║║║ ║  
% ╚═╝╚═╝╚═╝╩╝╚╝  ═╩╝╚═╝╚═╝╚═╝╩ ╩╚═╝╝╚╝ ╩  
%-----------------------------------------

\begin{document}
\pagenumbering{Roman} 

\setcounter{section}{0}
\include{preface/cover}

\include{preface/abstract}
\clearpage

%
% TABLE OF CONTENTS
%
\include{preface/tableofcontents}

%
% Abbildungsverzeichnis
%
\include{preface/listoffigures}

%
% Tabellenverzeichnis
%
\include{preface/listoftables}

%
% Listingverzeichnis
%
\include{preface/lstlistoflistings}


%--------------------------
% ╔═╗╦ ╦╔═╗╔═╗╔╦╗╔═╗╦═╗╔═╗ 
% ║  ╠═╣╠═╣╠═╝ ║ ║╣ ╠╦╝╚═╗ 
% ╚═╝╩ ╩╩ ╩╩   ╩ ╚═╝╩╚═╚═╝ 
%--------------------------

\pagenumbering{arabic} 
\setcounter{page}{1} 
\pagestyle{default}
%
% CHAPTER Einleitung
%
\chapter{Einleitung}
\label{chap:EINL}

In dem vierten Versuch der Versuchsreihe mit dem Titel  \textit{Aufbau eines einfachen Spracherkenners}, geht es um einen Spracherkenner und dessen Auswertung.
\newline
So werden wir einen einfachen Spracherkenner aufbauen, der beispielsweise zur Steuerung eines Staplers in einem Hochregallager dienen könnte.
\newline
Es reichen hierzu die Erkennung von vier einfachen Befehlen (Hoch, Tief, Links, Rechts). Der Aufbau  des Spracherkenners folgt dem in der Vorlesung beschriebenen Prinzip des Prototyp Klassifikators. 
\newline
\newline
Die zugehörigen Spektren werden mit der Windowing Methode berechnet. Zur Aufnahme der Befehle werden wir ein Mikrofon verwenden, dass mit der Soundkarte des Computers verbunden ist, über das Pythonpaket PyAudio.
\newline
Dieses ermöglicht eine bequeme Möglichkeit zum Auslesen der Soundkarte.
Zudem werden wir eine Triggerfunktion dem Ausleseprogramm hinzufügen, dass die Signale alle an der selben Stelle anfangen.
\newline
Daraufhin werden wir das Signal in ein Amplitudenspektrum anwenden und die Methode des Windowing anwenden.
Außerdem werden wir die Korrelationskoeffizienten anwenden.
Im Bild unten ist die Architektur des Spracherkenners zu erkennen.


\begin{figure}[H]
	\centering
	\includegraphics[width=.9\linewidth]{media/archi.png}
	\caption{Architektur des Spracherkenners}
	\label{img:Architektur des Spracherkenners}
\end{figure}
%
% CHAPTER Versuch 1
%
\chapter{Versuch 1}
\label{chap:VERSUCH_1}

\section{Fragestellung, Messprinzip, Aufbau, Messmittel}
\label{chap:VERSUCH_1_FRAGESTELLUNG}

\textbf{Fragestellung:}
\newline
In dem ersten Teil des Versuchs \textit{'Aufbau eines einfachen Spracherkenners'} werden wir einen Spracherkenner bauen.
Zudem erweitern wir den Spracherkenner mit einer Triggerfunktion.
Deshalb werden wir zuerst eine Testaufnahme machen.
Durch die entstandenen Numpy-Dateien bestimmen wir deren Amplitudenspektrum.
Im Anschluss werden wir die Methode des Windowing anwenden.
\newline
\newline
\textbf{Messprinzip:}
\newline
Im ersten Versuch starten wir mit einem Pythonskript. Das Mikrofon wird mit einem Klinkenstecker direkt an die Soundkarte des Computers verbunden.
Auf den Computern im Labor, ist das Paket PyAudio bereits in den IDE's integriert.
\newline
Mit dem geschriebenen Pythonskript können wir nun akustische Signale aufnehmen. 
Über das Objekt \textit{audiorecorder} haben wir Zugriff auf die Aufnahmefunktion der Soundkarte.
\newline
Das Signal speichern wir anschließend mittels \textit{numpy.save()}.
Danach sollen wir eine beliebige Spracheingabe aufnehmen und diese in einem Diagramm darstellen.
\newline
\newline
Im Anschluss sollen wir das Aufnahmeprogramm um eine Triggerfunktion erweitern, welche die Aufnahme erst ab einem gewissen Lautstärkepegel starten lässt.
\newline
So können wir sicher stellen, dass alle Aufnahmen den selben Startpunkt besitzen.
\newpage
Das Signal soll eine Dauer von einer Sekunde haben und die fehlenden Samples mit Nullen augefüllt werden.
\newline
\newline
Mit dem Code aus dem dritten Versuch können wir mit der Aufnahme das Amplitudenspektrum bestimmen.
Dies stellen wir grapisch dar.
Danach implementieren wir die Methode des Windowing. 
\newline
Diese werden wir jeweils in einer Länge von 512 Samples darstellen. Die Windows überlappen sich jeweils zur Hälfte.
Die einzelnen Windows werden wir mit der Gaußschen Fensterfunktion multiplizieren  die eine Fensterbreite von 4 Standartabweichungen hat.
\newline
Den ersten Versuch werden wir mit dem Amplitudenspektrum erneut überprüfen und so das Spektrum aus der letzten Aufgabe auf Korrektheit überprüfen.
\newline
\newline
\textbf{Aufbau:}
\newline
Das Mikrofon wird durch einen Klinkenstecker direkt an die Soundkarte des Computers verbunden.
Durch ein Pythonskript können wir nun Sprachaufnahmen machen.
\begin{figure}[H]
	\centering
	\includegraphics[width=.5\linewidth]{media/aufbau.jpg}
	\caption{Aufnahme mit Triggerfunktion des Wortes rechts}
	\label{img:Aufnahme mit Triggerfunktion des Wortes rechts}
\end{figure}
\newpage
\textbf{Messmittel:}
\item[NichtnummerierteAufzahlung]~\par
   \begin{itemize}
      \item Ein Mikrofon
      \item Ein Computer mit einer Python IDE
   \end{itemize}
\newpage
\section{Messwerte}
\label{chap:VERSUCH_1_MESSWERTE}
Mit einem Mikrofon das an den Computer angeschlossen ist, sehen wir die mit dem Pythonskript aufgenommene Sprachaufnahme des Wortes \textit{'Test'}, welche mit Python visualisiert wurde.
\newline
In der linken Darstellung ist die gesamte Dauer des empfangenen Signals, welche sich durch einen einfachen Plot des Signals darstellen lässt.
\newline
Im rechten Bild ist eine verkürzte Darstellung der Sprachaufnahme um das Signal besser zu erkennen welches durch \textit{plt.xlim(0, 35000)} auf eine Länge der x-Achse von 35000 limitiert wird.

\begin{figure}[hbt!]
\centering
	\begin{subfigure}{.5\textwidth}
  		\centering
 		 \includegraphics[width=.95\linewidth]{../data/img/testamp.png}
  		\caption{Sprachaufnahme des Wortes Test}
 		 \label{fig:sub1}
	\end{subfigure}%
	\begin{subfigure}{.5\textwidth}
  		\centering
 		 \includegraphics[width=.95\linewidth]{../data/img/testamp2.png}
  		\caption{Dieselbe Sprachaufnahme mit kurzer Zeitachse}
  		\label{fig:sub2}
	\end{subfigure}
	\caption{Die aufgenommene Sprachaufnahme visualisiert mit Python}
	\label{fig:test}
\end{figure}
Die Aufnahme wurde durch die Funktion \textit{p.open()} gestartet. 
\newline
Im Gegensatz zum zweiten Teil der Aufgabe wird hier bereits durch das Starten des Skriptes aufgenommen.
\newpage
Im folgenden sieht man die Sprachaufnahme des Wortes 'Rechts' mit der Triggerfunktion.
Die Aufnahme läuft auch hier bereits durch das Starten des Programmes.
\newline
Allerdings wird die Stelle bei der die Überschreitung des Schwellenwertes geschieht gespeichert und die tatsächliche Aufnahme beginnt erst ab dieser Stelle.
\newline
Durch die Triggerfunktion startet die Aufnahme also dadurch erst ab dem Schwellenwert.
\begin{figure}[H]
	\centering
	\includegraphics[width=.75\linewidth]{../data/img/rechtsamp.png}
	\caption{Aufnahme mit Triggerfunktion des Wortes rechts}
	\label{img:aua}
\end{figure}

\newpage
\section{Auswertung}
\label{chap:VERSUCH_1_AUSWERTUNG}
Auf der linken Seite ist das Amplitudenspektrum und auf der rechten Seite dasselbe Amplitudenspektrum allerdings nur bis zu einer Frequenz von 1000.
\newline
Diese wurden mit der folgenden Formel berechnet. 
\\~\\
f = $\dfrac{n}{M * \Delta t}$
\\~\\
\newline
Mit der Anwendung der Formel auf jedes eingegangene Signal erhalten wir den folgenden Plot:

\begin{figure}[H]
\centering
	\begin{subfigure}{.5\textwidth}
  		\centering
 		 \includegraphics[width=.95\linewidth]{../data/img/testspektrum1.png}
  		\caption{Amplitudenspektrum der Sprachaufnahme}
 		 \label{fig:sub1}
	\end{subfigure}%
	\begin{subfigure}{.5\textwidth}
  		\centering
 		 \includegraphics[width=.95\linewidth]{../data/img/testspektrum3.png}
  		\caption{Amplitudenspektrum mit geringerer Frequenz}
  		\label{fig:sub2}
	\end{subfigure}
	\caption{Das Amplitudenspektrum mit der dazugehörigen Frequenz}
	\label{fig:test}
\end{figure}
\newpage
Im folgenden sieht man eine Gaußsche Fensterfunktion mit der Fensterbreite von 4 Standardabweichungen.
\begin{figure}[H]
	\centering
	\includegraphics[width=.6\linewidth]{../data/img/gauss.png}
	\caption{Gausssche Fensterfunktion mit Standardabweichung 4}
	\label{img:Gausssche Fensterfunktion mit Standardabweichung 4}
\end{figure}
Im Anschluss zerlegt man das Signal in Abschnitte mit einer Länge von 512 Samples, die sich jeweils zur Hälfte überlappen sollen.
Das Signal wird mit dem entsprechenden Signal der Gaußschen Fensterfunktion multipliziert und daraufhin eine lokale Fouriertransformation durchgeführt.
Zu guter Letzt werden die Fouriertransformierten über alle Fenster gemittelt.

\begin{figure}[H]
	\centering
	\includegraphics[width=.6\linewidth]{../data/img/AlleRichtig.png}
	\caption{Aufnahme mit Triggerfunktion des Wortes rechts}
	\label{img:Aufnahme mit Triggerfunktion des Wortes rechts}
\end{figure}
\newpage
Das Windowing wird auf das Signal angewandt. Nachdem zum Beispiel bei dem Wort 'Hoch' über 170 Windows ausgegeben wurden als Plot, sieht der Explorer folgendermaßen aus.
\begin{figure}[H]
	\centering
	\includegraphics[width=.8\linewidth]{../data/img/window.png}
	\caption{Aufnahme mit Triggerfunktion des Wortes rechts}
	\label{img:Aufnahme mit Triggerfunktion des Wortes rechts}
\end{figure}
\newpage
\section{Interpretation}
\label{chap:VERSUCH_1_INTERPRETATION}
Bei der Betrachtung der Sprachaufnahme sehen wir die übliche Amplituden und Frequenzen bei der Aussprache des Wortes 'Test'.
In der Vergrößerung sind nochmals besser die Amplitudenausschläge zu sehen.
\newline 
\newline
In der Abbildung [\ref{img:aua}] sieht man die Sprachaufnahme des Wortes 'Rechts'.
Die Aufnahme wurde mit unserem Pythonskript gemacht, dass eine Triggerfunktion implementiert hat.
\newline
Ab einem bestimmten Schwellenwert beginnt die Aufnahme. 
\newline
Hier ist es schön zu sehen, dass die Aufnahme nicht mit Start des Skriptes beginnt, sondern erst wenn gesprochen wird. 
\newline 
\newline
Auf das Signal wurde daraufhin das Amplitudenspektrum berechnet. Durch dieses erfährt man die Frequenzen die in einem Wort stecken.
Der größte Ausschlag bei der Frequenz mit eienr Amplitude von über 0.8 ist die maximalste Amplitude. 
\newline
Die Frequenz liegt bei knapp über 300Hz.
Die x Achse wurde  auf die Hälfte gekürzt, da bei ein Amplitudenspektrum sich nach der Hälfte wiederholt.
\newline
\newline
Wenn man die einzelnen Windows die man durch das Windowing erhalten hat mit dem Gaußschen Fenster multipliziert und im Anschluss absolut fouriertransformiert, erhält man die gemittelten fouriertransfomierten Fenster.
\newline
Desto höher der absolute Ausschlag, desto höher ist die Amplitude im Window. Durch die senkrechten Ausschnitte sind steile Übergänge erzeugt
worden, die im ursprünglichen Signal nicht vorhanden sind. 
\newline
Steile Übergänge erzeugen jedoch viele hohe Frequenzen im Spektrum.
%
% CHAPTER Versuch 2
%
\chapter{Versuch 2}
\label{chap:VERSUCH_2}
\section{Fragestellung, Messprinzip, Aufbau, Messmittel}
\label{chap:VERSUCH_2_FRAGESTELLUNG}
Im zweiten Versuch müssen wir vier verschiedene Befehle jeweils fünf mal aufnehmen. 
\newline
Die Befehle die wir aufnehmen lauten:
\item[NichtnummerierteAufzahlung]~\par
   \begin{itemize}
      \item Links
      \item Rechts
      \item Hoch
      \item Tief
   \end{itemize}
Anhand der aufgenommenen Numpy Funktionen berechnen wir die Spektren mit der Windowing Methode.
Daraufhin berechnen wir noch die eigentlichen Referenzspektren.
\newline
Im Anschluss berechnen wir noch den Korrelationskoeffizienten nach Bravais-Pearson mit dem wir zwei verschiedene Spektren miteinander vergleichen können.
\newline
Dafür brauchen wir erneut vom selben und von einem anderen Sprecher Sprachaufnahmen.
\newline
Mit diesen testen wir die Routine an den Referenzspektren: 
\newline
Beim Vergleich identischer Spektren sollte die Korrelation 1 sein, bei verschiedenen Spektren nahe an 0.
\newline
Zudem sollen wir eine Fehler- und eine Detektionsrate angeben.

\section{Messwerte}
\label{chap:VERSUCH_2_MESSWERTE}
Die sämtlichen aufgenommenen Sprachaufnahmen in Form von Numpy-Dateien werden wir mit der Windowing Methode in Spektren berechnen.
Dabei wird erstmal das Spektrum mit der Gaußschen Fensterfunktion berechnet und multipliziert mit den einzelnen Windows.
\newline
Danach wird das Fenster absolut fouriertransformiert.
Daraus wiederrum wird der Durchschnitt berechnet. 
\begin{figure}[H]
	\centering
	\includegraphics[width=.7\linewidth]{media/folder.png}
	\caption{Sämtliche Signale mit Windowing berechnet}
	\label{img:Sämtliche Signale mit Windowing berechnet}
\end{figure}

\begin{figure}[H]
\centering
	\begin{subfigure}{.5\textwidth}
  		\centering
 		 \includegraphics[width=.95\linewidth]{../data/img/Versuch2/hoch/links1AlleRichtig.png}
  		\caption{Erste Sprachaufnahme von 'links'}
 		 \label{fig:sub1}
	\end{subfigure}%
	\begin{subfigure}{.5\textwidth}
  		\centering
 		 \includegraphics[width=.95\linewidth]{../data/img/Versuch2/hoch/links5AlleRichtig.png}
  		\caption{Fünfte Sprachaufnahme von 'links'}
  		\label{fig:sub2}
	\end{subfigure}
	\caption{Zwei verschiedene Sprachaufnahmen im Vergleich.}
	\label{fig:test}
\end{figure}
\newpage
\section{Auswertung}
\label{chap:VERSUCH_2_AUSWERTUNG}
Die Daten aus den Numpydateien werden in einzelne Windows zerlegt mit einer Länge von 512 Samples die sich jeweils zur Hälfte überlappen.
\newline
Die einzelnen Samples werden mit der Gaußschen Fensterfunktion multipliziert, welche eine Fensterbreite von 4 Standardabweichungen hat.
In jedem Fenster wird eine lokale Fouriertransformation durchgeführt.
Daraufhin werden die Fouriertransformierten über alle Fenster gemittelt.
\newline
Nachdem man sämtliche 4 Befehle und die jeweils fünf Beispiel dazu gemittelt hat, erhält man die folgenden Referenzspektren.
\begin{figure}[H]
\centering
	\begin{subfigure}{.5\textwidth}
  		\centering
 		 \includegraphics[width=.95\linewidth]{../data/img/Versuch2/2Averagelinks.png}
  		\caption{Person 1: Referenzspektrum 'Links'}
 		 \label{fig:sub1}
	\end{subfigure}%
	\begin{subfigure}{.5\textwidth}
  		\centering
 		 \includegraphics[width=.95\linewidth]{../data/img/Versuch2/2Averagerechts.png}
  		\caption{Person 1: Referenzspektrum 'Rechts'}
  		\label{fig:sub2}
	\end{subfigure}
\begin{subfigure}{.5\textwidth}
  		\centering
 		 \includegraphics[width=.95\linewidth]{../data/img/Versuch2/2Averagehoch.png}
  		\caption{Person 1: Referenzspektrum 'Hoch'}
 		 \label{fig:sub3}
	\end{subfigure}%
	\begin{subfigure}{.5\textwidth}
  		\centering
 		 \includegraphics[width=.95\linewidth]{../data/img/Versuch2/2Averagetief.png}
  		\caption{Person 1: Referenzspektrum 'Tief'}
  		\label{fig:sub4}
	\end{subfigure}
	\caption{Die vier Referenzspektren}
	\label{fig:2test}
\end{figure}
\newpage
Um auch die Richtigkeit der Pythonfunktion \textit{scipy.stats.pearsonr} zu gewähren, müssen wir einen Test machen.
\newline
Wenn genau zwei gleiche Signale mit dem Korrelationskoeffizienten berechnet werden, muss 1.0 herauskommen. 
\newline
Bei unterschiedlichen Signalen muss ein Wert errechnet werden der zwischen 1.0 und -1.0 liegt. 
\begin{table}[H]
	\centering\small
	\begin{tabular}{||c | c||}
		 \hline
		 \textbf{Spektrum} & \textbf{Korrelationskoeffizient} \\ [0.5ex] 
		 \hline
		 2 identische Messwerte & 1.0 \\ 
		 \hline
		 2 unterschiedliche Messwerte, dasselbe Wort & 0.6567 \\ 
		 \hline
	\end{tabular}
	\caption{Überprüfung auf Funktionalität des Bravais-Pearson Methode}
	\label{fig:Überprüfung auf Funktionalität des Bravais-Pearson Methode}
\end{table}
Wie erwartet erhalten wir einen Korrelationskoeffizienten von 1.0 bei den selben Messwerten.
Bei dem Korrelationskoeffizienten von 2 unterschiedlichen Messwerten, aber mit dem sprachlich gleichbedeutenden Input erhält man eine Annäherung und zwar 0.6567.
Damit wissen wir, dass unsere Pythonfunktion funktioniert.
\newpage
Hier werden zwei Eingabespektren beziehungsweise das Referenzspektrum mit einem weiteren Eingabespektrum mittels der Bravais-Pearson Methode verglichen auf ihre Korrelationskoeffizienten.
\newline
Wenn die Spektren identisch sind, sollte der Wert 1.0 ergeben. Bei ungleichen Spektren sollte dieser zwischen 1.0 und -1.0 sein.
Person 1 ist die Person, welche die Spracheeingabe gemacht hat für die Referenzspektren.

\begin{table}[H]
	\centering\small
	\begin{tabular}{||c | c | c||}
		 \hline
		 Spracheingabe & Person 1 & Person 2 \\ [0.5ex] 
		 \hline
		 \textbf{Hoch} & \textbf{0.5817} & \textbf{0.5496} \\ 
		 \hline
		 1 - Hoch & 0.6856 & 0.6283 \\ 
		 \hline
		 2 - Hoch & 0.5001 & 0.6000 \\ 
		 \hline
		 3 - Hoch & 0.5829 & 0.5600 \\ 
		 \hline
		 4 - Hoch & 0.5754 & 0.4902 \\ 
		 \hline
		 5 - Hoch & 0.5644 & 0.4691 \\ 
		 \hline
		 \textbf{Tief} & \textbf{0.7088} & \textbf{0.6172} \\
		 \hline
		 1 - Tief & 0.6682 & 0.5976 \\ 
		 \hline
		 2 - Tief & 0.7210 & 0.6520 \\ 
		 \hline
		 3 - Tief & 0.6947 &0.4947 \\ 
		 \hline
		 4 - Tief & 0.7069 &0.6098 \\ 
		 \hline
		 5 - Tief & 0.7531 & 0.7318 \\ 
		\hline
		\textbf{Links} & \textbf{0.6514} & \textbf{0.6084} \\
		\hline
		 1 - Links & 0.6708 & 0.5711 \\ 
		 \hline
		 2 - Links & 0.6451 & 0.5953 \\ 
		 \hline
		 3 - Links & 0.6273 & 0.5785 \\ 
		 \hline
		 4 - Links & 0.6472 & 0.6512 \\ 
		 \hline
		 5 - Links & 0.6666 &  0.6460 \\ 
		 \hline
		 \textbf{Rechts} & \textbf{0.5002} & \textbf{0.1779} \\
		 \hline
		 1 - Rechts & 0.5001 & - 0.1401 \\ 
		 \hline
		 2 - Rechts & 0.5402 & 0.2892 \\ 
		 \hline
		 3 - Rechts & 0.5032 & 0.2790 \\ 
		 \hline
		 4 - Rechts & 0.3895 & 0.2849 \\ 
		 \hline
		 5 - Rechts & 0.5682 & 0.1764 \\ 
		 \hline
	\end{tabular}
	\caption{Vergleich der berechneten Referenzspektren}
	\label{tab:tan}
\end{table}
\newpage

\begin{figure}[H]
\centering
	\begin{subfigure}{.5\textwidth}
  		\centering
 		 \includegraphics[width=.95\linewidth]{../data/img/Versuch2/2Averagerechts.png}
  		\caption{Person 1: Referenzspektrum 'Rechts'}
 		 \label{fig:sub1}
	\end{subfigure}%
	\begin{subfigure}{.5\textwidth}
  		\centering
 		 \includegraphics[width=.95\linewidth]{../data/img/Versuch2/Averagerechts.png}
  		\caption{Person 2: Referenzspektrum 'Rechts'}
  		\label{fig:sub2}
	\end{subfigure}
\begin{subfigure}{.5\textwidth}
  		\centering
 		 \includegraphics[width=.95\linewidth]{../data/img/Versuch2/2Averagehoch.png}
  		\caption{Person 1: Referenzspektrum 'Hoch'}
 		 \label{fig:sub3}
	\end{subfigure}%
	\begin{subfigure}{.5\textwidth}
  		\centering
 		 \includegraphics[width=.95\linewidth]{../data/img/Versuch2/Averagehoch.png}
  		\caption{ Person 2: Referenzspektrum 'Hoch'}
  		\label{fig:sub4}
	\end{subfigure}
	\caption{Die vier Referenzspektren mit den größten Unterschieden}
	\label{fig:test2}
\end{figure}
\newpage
\section{Interpretation}
\label{chap:VERSUCH_2_INTERPRETATION}
In Abbildung [\ref{fig:test}] sieht man, wie sich Sprachaufnahmen von der selben Person mit dem selben Wort unterscheiden können.
Im hinteren Teil von der linken Abbildung sieht man zum Schluss eine Steigung der Spannung. 
\newline
Diese ist entstanden durch ungefähr 8 bis 9 andere Gruppen im Raum, durch welche die Hintergrundgeräusche sehr laut waren.
Die anderen Geräusche unterscheiden sich nur geringfügig.
\newline
\newline
Nachdem wir die Signale ausgewertet haben, haben wir mit der Windowing Methode und der Bravais-Pearson den Korrelationskoeffizienten berechnet.
In der Abbildung [\ref{fig:2test}] sieht man die durchschnittlichen Spektren für eines der Worte (Hoch, Tief, Links, Rechts).
\newline
Dabei ähneln sich die Spektren der Wiederholungen der Worte sehr. Das Wort rechts hat ein sehr einprägsames Spektrum. 
\newline
Am Anfang geht das Signal sehr steil nach oben durch das Vokal 'e'.
Bei den Konsonanten 'ch' gibt es eine sehr große Verringerung. Zum Schluss bei 's' geht das Signal wieder nach oben.
So kann man gut die einzelnen Buchstaben des Wortes identifizieren.
\newline
\newline
Bei der Berechnung der Korrelationskoeffizienten nach der Bravais-Pearson Methode mit den Referenzspektren und der erneuten Aufnahme der Worte erhält man eine Lange Tabelle mit einigen Werten.
\newline
Die Bravais-Pearson Korrelationskoeffizientenberechnung hat sich gelohnt und als richtig erwiesen, da die Werte immer bei den Sprachaufnahmen der selben Person höher waren, als bei der fremden Person.
\newline
Bei den Wörtern 'Hoch', 'Tief' und 'Links' sieht man sehr gut die Ähnlichkeit der Korrelationskoeffizienten. Die Worte werden alle als das Wort des Referenzspektrums erkannt.
Sie ähneln sich nicht nur im Durchschnittswert sondern, es gibt auch nur wenige Ausreißer.
Die Aufnahmen des Wortes 'Hoch' schwächeln geringfügig bei den letzten beiden Aufnahmen der zweiten Person und bei 'Tief' die dritte Aufnahme, aber die Werte liegen dennoch im annehmbaren Bereich.
\newline
Bei 'Rechts' hingegen, gibt es klare und eindeutige Unterschiede. Die Worte werden hier vom zweiten Sprecher nicht richtig zugehörig zum Referenzspektrum erkannt.
Der größte Unterschied liegt bei der ersten Aufnahme des Wortes 'Rechts'.
Zu sehen ist hier ein Unterschied von 0.6401, was sehr viel ist wenn man bedenkt, dass die Korrelationskoeffizienten eine Spanne von 1.0 bis -1.0 haben.
\newpage
Hier [\ref{fig:test2}] kann nochmal gut einzelne Spektren begutachten, die das selbe Wort von unterschiedlichen Menschen darstellen.
Vorallem, dass der zweite Sprecher definitiv näher am Mikrofon beziehungsweise lauter gesprochen hat.
\newline
Die Amplituden liegen im zwei- bis sogar dreifachen Bereich und haben ein sehr viel unruhigeres Spektrum.
Daher kommen auch die so unterschiedlichen Werte in der Tabelle [\ref{tab:tan}].
Vorallem bei den ersten beiden Spektren in [\ref{fig:test2}]  sieht man woher dieser große Unterschied zu den Referenzspektren führt.

%
% CHAPTER Anhang
%
\renewcommand\thesection{A.\arabic{section}}
\renewcommand\thesubsection{\thesection.\arabic{subsection}}

\chapter*{Anhang}
\label{chap:APPENDIX}
\addcontentsline{toc}{chapter}{Anhang}
%\setcounter{chapter}{0}
\addtocounter{chapter}{1}
\setcounter{section}{0}

\section{Quellcode}
\label{chap:APPENDIX_SOURCECODE}

\subsection{Quellcode Versuch 1}
\label{chap:APPENDIX_SOURCECODE_V1}
\lstinputlisting[style=PYTHON, frame=single, caption=Einlesen der Sprachaufnahme und ablegen des Signals in eine Numpy Datei, captionpos=b, label=lst:APPENDIX_SOURCECODE_AREA1]{../task1.1.py}
\newpage
\lstinputlisting[style=PYTHON, frame=single, caption=Einlesen einer Sprachaufnahme mit Aktivierung durch Triggerung, captionpos=b, label=lst:APPENDIX_SOURCECODE_AREA2]{../task1.2.py}
\newpage
\lstinputlisting[style=PYTHON, frame=single, caption=Amplitudenspektrum und Ausgabe von Plots, captionpos=b, label=lst:APPENDIX_SOURCECODE_AREA3]{../task1.3.py}
\newpage
\lstinputlisting[style=PYTHON, frame=single, caption=Windowing und Ausgabe von Plots bzw. Windows, captionpos=b, label=lst:APPENDIX_SOURCECODE_AREA4]{../task1.4.py}
\newpage
\subsection{Quellcode Versuch 2}
\label{chap:APPENDIX_SOURCECODE_V2}
\lstinputlisting[style=PYTHON, frame=single, caption=Windowing und Mittelung der Spektren, captionpos=b, label=lst:APPENDIX_SOURCECODE_AREA5]{../task2.1.py}
\newpage
\lstinputlisting[style=PYTHON, frame=single, caption=Windowing und Bravais-Pearson Methode, captionpos=b, label=lst:APPENDIX_SOURCECODE_AREA6]{../task2.2.py}
\lstinputlisting[style=PYTHON, frame=single, caption=Bravais-Pearson Methode mit Ausgabe der Korrelation, captionpos=b, label=lst:APPENDIX_SOURCECODE_AREA7]{../task2.3.py}

%
% Literaturverzeichnis
%
\include{appendix/bibliography}

\end{document}
%------------------------------------
% ╔═╗╔╗╔╔╦╗  ╔╦╗╔═╗╔═╗╦ ╦╔╦╗╔═╗╔╗╔╔╦╗
% ║╣ ║║║ ║║   ║║║ ║║  ║ ║║║║║╣ ║║║ ║ 
% ╚═╝╝╚╝═╩╝  ═╩╝╚═╝╚═╝╚═╝╩ ╩╚═╝╝╚╝ ╩ 
%------------------------------------